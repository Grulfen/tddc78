\documentclass[a4paper, 12pt]{article}
\usepackage{graphicx}
\usepackage{fancyhdr}
\usepackage{listings}

\lstset{language=c++,
        basicstyle=\ttfamily\scriptsize,
        numbers=left}

\pagestyle{fancy}
\lhead{TDDC78 Lab 4}
\rhead{Hallberg, Svensk}

\begin{document}

\title{TDDC78 Lab 4\\
        Particle Simulation - MPI}
\author{Christopher Hallberg \\
        Gustav Svensk}
\maketitle

\thispagestyle{empty}

\newpage
\setcounter{page}{1}
\tableofcontents
\newpage

\section{Description}
This section provides a high level overview of the parallel program.
The structure of the program is:
\begin{itemize}
\item Setup
\item Particle generation
\item Simulation
\item Reduction
\end{itemize}
Our implementation is restricted to only run on a square number of processes,
this is to make it easier to split the particles among the processes. In our case
each process takes care of a smaller square and initializes particles inside.

We used vectors to store the particles to be able to benefit from cache effects,
and to avoid memory allocation and deallocation in the critical loop.
Using vectors also makes it easier to communicate the particles between processes.

\subsection{Setup}
In the setup phase the root processor broadcasts the number of time steps and
the number of particles in the simulation to every process. The communication
grid is setup using MPI\_Cart\_create and each process gets the id of its
neighbours via MPI\_Cart\_shift. The ids of the neighbours are saved in an
array. If a process does not have a neighbour the special id MPI\_PROC\_NULL is
saved instead. Each process checks if it has a wall on any of its sides. 
Each process has its own local coordinates.

\subsection{Particle generation}
In this phase each process generates as many particles as indicated by the
command line argument num\_particles. The particles position and velocity is
randomly generated. The function \textit{generate\_particles} is used to generate
the particles.
\subsection{Simulation}
The simulation phase contains a for-loop over the time steps. Inside the loop
each particle is checked for collisions with other particles, wall collisions,
whether the particle should be passed to another process and finally the
particles are communicated between the processes.

The details of the collision detection and handling can be seen in section
\ref{sec:src}.

If a particle is outside of the square it has either collided with a wall or
will pass over to another square. If it has a wall on that side the function
\textit{wall\_collide} is used to calculate the momentum and move the particle.
If there is no wall the particle is moved to a communication buffer and in the
end of the loop all the particles in the communication buffer are sent to the
corresponding neighbour.

When communicating between the processes the neighbour array is used. If a
process uses MPI\_Send to the special id MPI\_PROC\_NULL it immediately returns,
using this the communication can be done with a simple loop without any edge cases
for processes in the corners or edges of the box.

\subsection{Reduction}
In the reduction phase the local momentum calculated in all the processes are
summed in the root process using MPI\_Reduce. The pressure is then calculated
by dividing the momentum by the number of time steps and the wall length of the
box.

\section{Verification}
All the figures can be found in appendix \ref{sec:fig}.

The goal of this lab is to verify the gas law $pV = nRT$.
By rewriting the equation we get: $p = \frac{nRT}{V}$ and we see that the
pressure should be linear with the number of particles and the temperature and
inversely proportional to the volume, or in our case, the area.

In figure \ref{fig:n} we can see that the pressure is linear with the number of
particles. In figure \ref{fig:V} the pressure is plotted against one over the
area and we get a straight line, therefore the pressure is inversely
proportional to the area.

The kinetic temperature is proportional to the square of the velocity of the
particles. So by changing the maximal initial velocity of the particles we can
control the temperature. We can now write $p \propto \frac{nR(mv^2)}{2V}$ and
$\sqrt{p} \propto v$ if the other variables are constant. Figure \ref{fig:T}
verifies this.

These results verify the gas law.

\section{Performance}
\section{Source}
\label{sec:src}
The most interesting part of the code is the collision detection and handling of
the particles. The commented source can be seen below.
\begin{lstlisting}
        // Check for collisions and move particles
        for(vector<pcord_t>::iterator a = particles.begin(); a != particles.end() - 1; a++){
            for(vector<pcord_t>::iterator b = a+1; b != particles.end(); b++){
                collision_time = collide(&(*a),&(*b));
                // Collision
                if(collision_time != -1){
                    // collided, move particles
                    interact(&(*a), &(*b), collision_time);

                    // swap collided particle with next particle and skip to
                    // next iteration
                    swap(*(a+1), *(b));
                    a++;
                    break;
                }
            }
            if(collision_time != -1){
                if(a == particles.end())
                    // prevent errors if second last particle collides with last
                    // particle
                    break;
            } else {
                // if no collision move particle
                feuler(&(*a), STEP_SIZE);
            }
        }
\end{lstlisting}

Every particle is compared to all the particles to the right of itself for
collisions. The function \textit{collide} is used for determining the
collisions. If there is a collision both the particles are updated using
\textit{interact} and the momentum is added to the local momentum saved in the
variable \textit{l\_pressure}. The second particle is then swapped with the
particle next to the first one and the iterator is increased one step. This way
every particle is only checked once. If there is no collision the particle is
moved using \textit{feuler}.

\newpage
\appendix
\section{Figures}
\label{sec:fig}
\begin{figure}[hb]
        \centering
        \includegraphics[width=0.8\textwidth]{particles.png}
        \caption{Pressure vs number of particles}
        \label{fig:n}
\end{figure}
\begin{figure}[hb]
        \centering
        \includegraphics[width=0.8\textwidth]{area.png}
        \caption{Pressure vs $\frac{1}{area}$}
        \label{fig:V}
\end{figure}
\begin{figure}[hb]
        \centering
        \includegraphics[width=0.8\textwidth]{temperature.png}
        \caption{$\sqrt{Pressure}$ vs temperature}
        \label{fig:T}
\end{figure}

\end{document}
