\documentclass[a4paper, 12pt]{article}
\usepackage{graphicx}
\usepackage{fancyhdr}
\usepackage{listings}

\lstset{language=fortran,
        basicstyle=\ttfamily\scriptsize,
        numbers=left}

\pagestyle{fancy}
\lhead{TDDC78 Lab 3}
\rhead{Hallberg, Svensk}

\begin{document}

\title{TDDC78 Lab 3\\
        Heat Equation - OpenMP}
\author{Christopher Hallberg \\
        Gustav Svensk}
\maketitle

\thispagestyle{empty}

\newpage
\setcounter{page}{1}
\tableofcontents
\newpage

\section{Description}
The parallel implementation is based on the serial Fortran90 source code
provided. When running the executable the number of threads is given as
an argument, i.e. "./lab3 2" for two threads.

The outer loop is left un-parallelized while the inner loop is run on the
available threads. Since the memory organization in Fortran is opposite the one
in C, we now divide the work so that one thread always will work on entire
columns in order to utilize the cache properly. The columns are divided equally
among the threads, with the last thread taking any remainder from the integer
division. To prevent use of a value that has already been updated by another
thread in the same iteration temporary columns are saved by each thread.

Because the source code for this laboration is quite compact the
calculation kernel in its entirety can be found in section \ref{sec:src}.

\section{Performance}
The execution time for the whole outer loop was measured using the omp\_get\_time() function.
The result can be seen in figure \ref{fig:omp}.

When running with few threads the speedup is very good which is to be expected
since there is little communication between threads. The execution time flattens
out at around 12 threads and then increases.


\begin{figure}[h]
        \centering
        \includegraphics[width=0.8\textwidth]{omp.png}
        \caption{Execution times}
        \label{fig:omp}
\end{figure}

\newpage

\section{Source}
\label{sec:src}
The following listing shows the code for the outer loop.

\begin{lstlisting}
  do k=1,maxiter
     
     error=0.0D0
     !$OMP parallel default(private) shared(T,k) reduction(max:error)
     myid = omp_get_thread_num()
     team_size = omp_get_num_threads()
     even_cols = n/team_size;

     tmp_left=T(1:n,myid*even_cols)

     if(myid == team_size-1) then
        tmp_last=T(1:n,n+1)
        my_cols = even_cols + mod(n, team_size)
     else 
        my_cols = even_cols
        tmp_last = T(1:n,(myid+1)*even_cols+1)
     end if

     !$OMP BARRIER
     do j=myid*even_cols+1,myid*even_cols+my_cols

        tmp_mid=T(1:n,j)
        if(j == (myid+1)*my_cols) then
          tmp_right = tmp_last
        else
          tmp_right=T(1:n,j+1)
        end if
        T(1:n,j)=(T(0:n-1,j)+T(2:n+1,j)+tmp_right+tmp_left)/4.0D0
        error=max(error,maxval(abs(tmp_mid-T(1:n,j))))
        tmp_left=tmp_mid
     end do
     !$OMP end parallel
     
     if (error<tol) then
        exit
     end if
     
  end do
\end{lstlisting}


Going through the loop step by step, the first thing that is happening in each
iteration is that the error is reset. The OMP parallel directive is used around
everything besides checking the stopping criteria. Reduction with the max
function is used for computing the error. After the OMP parallel directive the
work is divided among the threads and the columns directly to the left and right
of a threads area is saved in temporary variables. In order to make sure that
all threads have saved these columns, a barrier is used before the inner loop
starts. The calculations done in the loop is almost the same as in the serial
code, but with modification for the temporary variables. An example is found on
line number 23 in the listing; if a thread is working on its
last column it needs to use the temp\_last column assigned outside of the inner
loop because it belongs to another thread. 

% \newpage
% \appendix
% \section{Figures}
% \label{sec:fig}
% \begin{figure}[h]
%         \centering
%         \includegraphics[width=0.8\textwidth]{omp.png}
%         \caption{Execution times}
%         \label{fig:omp}
% \end{figure}

\end{document}
