\documentclass[a4paper, 12pt]{article}
\usepackage{graphicx}
\usepackage{fancyhdr}
\usepackage{listings}

\lstset{language=c,
        basicstyle=\ttfamily\scriptsize}

\pagestyle{fancy}
\lhead{TDDC78 Lab 2}
\rhead{Hallberg, Svensk}

\begin{document}

\title{TDDC78 Lab 2\\
        Image filter - pthreads}
\author{Christopher Hallberg \\
        Gustav Svensk}
\maketitle

\thispagestyle{empty}

\newpage
\setcounter{page}{1}
\tableofcontents
\newpage

\section{Averaging Filter}
In the averaging filter each pixel gets its value according to a weighted average
of neighbouring pixels in the shape of a rectangle with a user specified radius. 
Our implementation uses the given Gaussian distribution.

\subsection{Description}
\label{sec:desc}
This section provides a high level overview of the parallel program. 
The code is of the same structure as the one provided on the course homepage,
but with added functionality for pthreads.

When the program is started the main thread allocates memory for the input image
and the resulting image and then reads the input file and calls the function
calculating the Gaussian coefficients. As in the averaging filter from lab one
the image is divided so that each thread will work on a continuous set of rows,
where the last thread will take the remaining part if the number of rows is not
evenly divisible by the number of threads.

A struct is used to store and pass all information that each thread needs. This
include source and destination pointers to their respective part of the image,
information about sizes and filtering parameters. The main thread loops over
the number of threads, spawning them after setting the appropriate parameters.
As in the previous lab the filter is divided into an x-part and a y-part. This
means that the threads have to be created and joined two times. Using two
pointers, one for source and one for destination, helps to avoid data hazards
when accessing image data outside of a thread's designated part. The program is
terminated after the main thread has written the image back to file.
 

\subsection{Performance}
The execution time is measured using the clock\_gettime() function. It is
started after the main thread has computed the Gaussian coefficients 
and is ended before writing the output file. 

One node (16 cores) was used during the simulation. Each of the images was
filtered with an increasing number of threads and using four different radii
(4, 8, 16 and 32). The results can be seen in the figures
\ref{fig:blur_4}, \ref{fig:blur_8}, \ref{fig:blur_16} and \ref{fig:blur_32} in
appendix \ref{sec:fig_aver}.

All the data sets follow the same general pattern but with different execution
times. The greatest performance improvement, in absolute numbers,
is found when increasing the number of threads from one to two. Then the execution 
time is halved, giving an optimal improvement. In relative values this continues
to hold even for some additional number of threads. However, the curves deviate
from the optimal case with increasing number of threads. The deviation is
smaller for larger problem sizes. For the largest image with a radius of 32 
going from one thread to eight threads gives a speed-up of about 7.6, while
going from one thread to 16 gives a speed-up of about 14. The same numbers for
the smallest images with radius of four are 7.3 and 11.6, respectively. 

At 17 threads the number of software threads exceeds the number of hardware
threads available and a sudden decrease of performance can be seen then. Adding
even more threads will yield slightly better performance compared to that of 17.
One probable reason for this can be that when using more software threads than
hardware threads the low overhead context switch can hide effects of a thread
having to wait, for example because of a cache miss.


\subsection{Source}

The source code can be found in the attached source code files.
The functionality of the code is described in section \ref{sec:desc} and in the
comments in the source code.


\section{Threshold Filter}
The threshold filter computes the average intensity of the image and then use
this as a threshold value.


\subsection{Description}
\label{sec:desc2}
The parallelized code keeps the basic structure of the example code provided on
the course homepage and the parallelization is done in a straightforward
fashion. The pixels are divided horizontally to use the data locality, but it is
the pixels and not the lines that are split (not the same as in the averaging
filter). The last thread takes care of the part remaining after the integer
division.

The filter function has been split into two separate functions, one calculating
the threshold value and one doing the filtering.

When starting the execution the main thread allocates memory for the image and
then reads it. The main thread then calculates how many pixels each thread
should filter and then spawns the threads that calculate the threshold value for
their part of the image. After that the main threads joins with all the threads
and add their threshold value to the global threshold value.

New threads are then spawned to filter their part of the image. The main thread
then joins with all the threads and the filtering is finished. The filtered
image is then written to a file.

\subsection{Performance}

The execution time is measured using the clock\_gettime() function. It is
started after the main thread has read the input file and is ended before
writing the output file.

The filter is run with 16 processors

The results can be found in figure \ref{fig:thres} in appendix
\ref{sec:fig_thres}. As in the averaging filter the performance improvement is
greater in the beginning. We can see that the execution time decreases as we
increase the number of threads. When using 17 threads the execution time
increases, this is because we use 16 cores and one hardware thread per core. So
when we have 17 threads there are more threads than hardware threads.

\subsection{Source}

The most interesting part of the source is where the pixels are divided among
the threads, the data structs are initialized and the threads are spawned. 
This part can be seen below.
\begin{lstlisting}
pix_p_thread = xsize*ysize/NUM_THREADS;

for(i=0;i<NUM_THREADS;i++){
        pdata[i] = malloc(sizeof(pixel_n));
        pdata[i]->src = src + i*pix_p_thread;
        pdata[i]->n = pix_p_thread;
        if(i == NUM_THREADS - 1){
                pdata[i]->n += xsize*ysize - pix_p_thread*NUM_THREADS;
        }
        pdata[i]->sum = malloc(sizeof(int));
        *(pdata[i]->sum) = 0;
        ret = pthread_create(&threads[i], NULL, threshold, (void*)pdata[i]);
        if(ret) {
                fprintf(stderr, "ERROR creating thread\n");
                free(pdata[i]->sum);
                free(pdata[i]);
                free(src);
                exit(-1);
        }
}
\end{lstlisting}

The rest of the source code can be found in the attached source code files.
The functionality of the code is described in section \ref{sec:desc2} and in the
comments in the source code.

\newpage
\appendix
\section{Figures - Averaging}
\label{sec:fig_aver}

\begin{figure}[hb]
        \centering
        \includegraphics[width=0.8\textwidth]{blur_4.png}
        \caption{Averaging filter with radius 4}
        \label{fig:blur_4}
\end{figure}
\begin{figure}[hb]
        \centering
        \includegraphics[width=0.8\textwidth]{blur_8.png}
        \caption{Averaging filter with radius 8}
        \label{fig:blur_8}
\end{figure}
\begin{figure}[hb]
        \centering
        \includegraphics[width=0.8\textwidth]{blur_16.png}
        \caption{Averaging filter with radius 16}
        \label{fig:blur_16}
\end{figure}
\begin{figure}[hb]
        \centering
        \includegraphics[width=0.8\textwidth]{blur_32.png}
        \caption{Averaging filter with radius 32}
        \label{fig:blur_32}
\end{figure}
\clearpage

\section{Figures - Threshold}
\label{sec:fig_thres}

\begin{figure}[h]
        \centering
        \includegraphics[width=0.8\textwidth]{threshold.png}
        \caption{Threshold filter}
        \label{fig:thres}
\end{figure}

\end{document}
