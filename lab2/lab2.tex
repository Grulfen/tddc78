\documentclass[a4paper, 12pt]{article}
\usepackage{graphicx}
\usepackage{fancyhdr}
\usepackage{listings}

\lstset{language=c,
        basicstyle=\ttfamily\scriptsize}

\pagestyle{fancy}
\lhead{TDDC78 Lab 2}
\rhead{Hallberg, Svensk}

\begin{document}

\title{TDDC78 Lab 2\\
        Image filter - pthreads}
\author{Christopher Hallberg \\
        Gustav Svensk}
\maketitle

\thispagestyle{empty}

\newpage
\setcounter{page}{1}
\tableofcontents
\newpage

\section{Averaging Filter}
In the averaging filter each pixel gets its value according to a weighted average
of neighbouring pixels in the shape of a rectangle with a user specified radius. 
Our implementation uses the given Gaussian distribution.

\subsection{Description}
\label{sec:desc}
This section provides a high level overview of the parallel program. 
The code is of the same structure as the one provided on the course homepage,
but with added functionality for pthreads.

When the program is started the main thread allocates memory for the input image
and the resulting image and then reads the input file and calls the function
calculating the Gaussian coefficients. 

%%
% OLD REPORT
%%
In order to utilize the cache functionality as much as possible we split the image
horizontally in equally sized parts. If the y-size of the image is not divisible
by the number of processors used the last processor will be assigned the
additional lines. Each processor will allocate local memory for their part of the
image plus extra shadow pixels needed to filter along the y-axis. The memory
needed for these shadow pixels will correspond to two times the width of the
image times the radius of the filtering rectangle. The root processor then 
scatters the image to every processor.  

The filter function provided have been split into two separate functions, the
first filtering along the x-axis and the other filtering along the y-axis. Some
small changes had to be made to copy memory and handle the shadow pixels. The
y-filter have two additional arguments specifying if the process will need
information about shadow pixels from another process or not. 

If there is only one process it will just call the two filter functions. If
there are more than one process the root process first sends the part of the
image left over after the integer division to the last process. All processes
filter along the x-axis which require no message passing between them.
The required shadow pixels are then passed, starting by first sending "upward"
then "downward". The y-filter is called by each process with the arguments 
corresponding to the position in the image. 

Passing the filtered image back to the root process is done by the last process 
first sending the filtered left over part from the integer division. The main
part of the image is passed back using the gather function.
The root then writes the image back to a file.
 

\subsection{Performance}
The execution time is measured using the MPI\_Wtime() function. It is started
after the root processor has calculated the Gaussian
coefficients and is ended before the root processor writes the output file.
The execution times are measured by running the program nine times and removing the
first result to eliminate cache effects. The results can be seen in the figures
\ref{fig:blur_4}, \ref{fig:blur_8}, \ref{fig:blur_16} and \ref{fig:blur_32} in
appendix \ref{sec:fig_aver}.

The first thing to note is that the greatest improvement due to parallelization
can be seen when increasing the number of processes when the number of processes
is low. When using many processes the improvement of adding more processes is
very small. The curves flattens out at around 35 processes, this is due to the
overhead of more processes is as big as the parallelization gain.

There is a strange peak at seven processes that we do not know the cause of. We don't
know if this is due to out algorithm or due to the scheduler at NSC.
The percental performance increase is very similar across image sizes and radii.
The performance increase for images 2, 3 and 4 is very big from one process to
two and three to four but not as big from two to three. This is not true for image 1. This is
because the number of rows in image 1 is not divisible by two or 4.

\subsection{Source}

The source code can be found in the attached source code files.
The functionality of the code is described in section \ref{sec:desc} and in the
comments in the source code.


\section{Threshold Filter}
The threshold filter computes the average intensity of the image and then use
this as a threshold value.


\subsection{Description}
\label{sec:desc2}
The parallelized code keeps the basic structure of the example code provided on
the course homepage and the parallelization is done in a straightforward
fashion. The pixels are divided horizontally to use the data locality, but it is
the pixels and not the lines that are split (not the same as in the averaging
filter). The last thread takes care of the part remaining after the integer
division.

The filter function has been split into two separate functions, one calculating
the threshold value and one doing the filtering.

When starting the execution the main thread allocates memory for the image and
then reads it. The main thread then calculates how many pixels each thread
should filter and then spawns the threads that calculate the threshold value for
their part of the image. After that the main threads joins with all the threads
and add their threshold value to the global threshold value.

New threads are then spawned to filter their part of the image. The main thread
then joins with all the threads and the filtering is finished. The filtered
image is then written to a file.

\subsection{Performance}

The execution time is measured using the clock\_gettime() function. It is
started after the main thread has read the input file and is ended before
writing the output file.

The filter is run with 16 processors

%%
% OLD REPORT
%%

The results can be found in figure \ref{fig:thres} in appendix
\ref{sec:fig_thres}. As in the averaging filter the performance improvement is
greater in the beginning. We can see that the execution time decreases as we
increase the number of threads. When using 17 threads the execution time
increases, this is because we use 16 cores and one hardware thread per core. So
then we have 17 threads there are more threads than hardware threads.

\subsection{Source}

The most interesting part of the source is where the pixels are divided among
the threads, the data structs are initialized and the threads are spawned. 
This part can be seen below.
\begin{lstlisting}
pix_p_thread = xsize*ysize/NUM_THREADS;

for(i=0;i<NUM_THREADS;i++){
        pdata[i] = malloc(sizeof(pixel_n));
        pdata[i]->src = src + i*pix_p_thread;
        pdata[i]->n = pix_p_thread;
        if(i == NUM_THREADS - 1){
                pdata[i]->n += xsize*ysize - pix_p_thread*NUM_THREADS;
        }
        pdata[i]->sum = malloc(sizeof(int));
        *(pdata[i]->sum) = 0;
        ret = pthread_create(&threads[i], NULL, threshold, (void*)pdata[i]);
        if(ret) {
                fprintf(stderr, "ERROR creating thread\n");
                free(pdata[i]->sum);
                free(pdata[i]);
                free(src);
                exit(-1);
        }
}
\end{lstlisting}

The rest of the source code can be found in the attached source code files.
The functionality of the code is described in section \ref{sec:desc2} and in the
comments in the source code.

\newpage
\appendix
\section{Figures - Averaging}
\label{sec:fig_aver}

\begin{figure}[hb]
        \centering
        \includegraphics[width=0.8\textwidth]{blur_4.png}
        \caption{Averaging filter with radius 4}
        \label{fig:blur_4}
\end{figure}
\begin{figure}[hb]
        \centering
        \includegraphics[width=0.8\textwidth]{blur_8.png}
        \caption{Averaging filter with radius 8}
        \label{fig:blur_8}
\end{figure}
\begin{figure}[hb]
        \centering
        \includegraphics[width=0.8\textwidth]{blur_16.png}
        \caption{Averaging filter with radius 16}
        \label{fig:blur_16}
\end{figure}
\begin{figure}[hb]
        \centering
        \includegraphics[width=0.8\textwidth]{blur_32.png}
        \caption{Averaging filter with radius 32}
        \label{fig:blur_32}
\end{figure}
\clearpage
\section{Figures - Threshold}
\label{sec:fig_thres}

\begin{figure}[h]
        \centering
        \includegraphics[width=0.8\textwidth]{threshold.png}
        \caption{Threshold filter}
        \label{fig:thres}
\end{figure}

\end{document}
